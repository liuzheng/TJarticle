%%% !Mode\dots "XeLaTeX:UTF-8"
\documentclass[a4paper,12pt]{article}
\input{setup/package}
% 页眉设置
\geometry{left=2.5cm,right=2.5cm,bottom=2.5cm,top=3cm}

\begin{document}
\input{setup/format}
%\cnarticle			% 中文论文环境,默认是英文的
%\renewcommand{\thefigure}{\arabic{section}-\arabic{figure}}       % 使图编号为 7-1 的格式 
%\renewcommand{\thesubfigure}{\alph{subfigure})}                   % 使子图编号为 a) 的格式
%\renewcommand{\thesubtable}{(\alph{subtable})}                    % 使子表编号为 (a) 的格式
%\renewcommand{\thetable}{\arabic{section}-\arabic{table}}         % 使表编号为 7-1 的格式
%\renewcommand{\theequation}{\arabic{section}-\arabic{equation}}   % 使公式编号为 7-1 的格式
\yemeiclean  % 清理页眉,注释掉会显示页眉

\begin{center}
\sihao    城南旧事
\wuhao

林海音

linghaiyin@tongji.edu.cn

\end{center}
\begin{multicols}{2}


\section*{\hfill 摘要 \hfill}
太阳从大玻璃窗透进来\cite{CK1}这里用的是$\backslash$ cite\{xx\},是latex默认的,照到大白纸糊的墙上\cankao{CK2},这里是用的我重写的$\backslash$ cankao\{xxx\},自己看区别吧。照到三屉桌上,照到我的小床上来了。我醒了,还躺在床上,看那道太阳光里飞舞着的许多小小的,小小的尘埃。宋妈过来掸窗台,掸桌子,随着鸡毛掸子的舞动,那道阳光里的尘埃加多了,飞舞得更热闹了,我赶忙拉起被来蒙住脸,是怕尘埃把我呛得咳嗽。

宋妈的鸡毛掸子轮到来掸我的小床了,小床上的棱棱角角她都掸到了,掸子把儿碰在床栏上,格格地响,我想骂她,但她倒先说话了:


\section{Introduction}

“还没睡够哪!”说着,她把我的被大掀开来,我穿着绒褂裤的身体整个露在被外,立刻就打了两个喷嚏。她强迫我起来,给我穿衣服。印花斜纹布的棉袄棉裤,都是新做的,棉裤筒多可笑,可以直立放在那里,就知道那棉花够多厚了。

妈正坐在炉子边梳头,倾着身子,一大把头发从后脖子顺过来,她就用篦子篦呀篦呀的,炉上是一瓶玫瑰色的发油,天气冷,油凝住了,总要放在炉子上化一化才能擦。


\begin{lstlisting}[language=TeX,numbers=none,frame=lrtb,keywords={begin},label=lstlisting,caption=多栏]
\begin{multicols}{2}
xxxxxx
\end{multicols}
\end{lstlisting}

\end{multicols}
\section{公式?!}
\$ Equation \$ = $ Equationg $

\href{http://www.sciweavers.org/free-online-latex-equation-editor}{手点}


\href{http://webdemo.visionobjects.com/home.html;jsessionid=1mk45ecrnd2zk1l5x938b6sd4y?locale=zh_CN#equation}{手写}

\input{body/equation}



\section{图表}
\subsection{插图}

\begin{figure}[H]
\includegraphics[width=8cm]{figure1.jpg}
\caption{Shubert Function}
\end{figure}

\begin{lstlisting}[language=TeX,numbers=none,frame=lrtb,keywords={begin},label=Gamma,caption=Gamma]
\begin{figure}[H]
\includegraphics[width=8cm]{figure1.jpg}
\caption{Shubert Function}
\end{figure}
\end{lstlisting}

\begin{figure}[H]
\centering
\includegraphics[width=8cm]{figure1.jpg}
\caption{Shubert Function}
\end{figure}

\begin{lstlisting}[language=TeX,numbers=none,frame=lrtb,keywords={begin},label=Gamma,caption=Gamma]
\begin{figure}[H]
\centering
\includegraphics[width=8cm]{figure1.jpg}
\caption{Shubert Function}
\end{figure}
\end{lstlisting}
\begin{lstlisting}[language=TeX,numbers=none,frame=lrtb,keywords={begin},label=Gamma,caption=Gamma]
\begin{figure}[H]
	\begin{center}
	\includegraphics[width=8cm]{figure1.jpg}
	\caption{Shubert Function}
	\end{center}
\end{figure}
\end{lstlisting}


\begin{figure}[H]
\centering
\subfigure[恩恩]{
\includegraphics[width=4cm]{figure1.jpg}
\includegraphics[width=4cm]{figure21.jpg}
}
\subfigure[呵呵]{
\includegraphics[width=4cm]{figure1.jpg}
\includegraphics[width=4cm]{figure1.jpg}
}
\caption{The trace of NO.2 player , NO.5 player and ball from cycle 238 to cycle 258}
\end{figure}

\begin{lstlisting}[language=TeX,numbers=none,frame=lrtb,keywords={begin},label=Gamma,caption=Gamma]
\begin{figure}[H]
\centering
\subfigure[ 恩恩]{
\includegraphics[width=4cm]{figure1.jpg}
\includegraphics[width=4cm]{figure21.jpg}
}
\subfigure[ 呵呵]{
\includegraphics[width=4cm]{figure1.jpg}
\includegraphics[width=4cm]{figure1.jpg}
}
\caption{The trace of NO.2 player , NO.5 player and ball from cycle 238 to cycle 258}
\end{figure}
\end{lstlisting}

\begin{figure}[H]
\centering
\subfigure[the first subfigure]{
\begin{minipage}[b]{0.2\textwidth}
\includegraphics[width=1\textwidth]{figure1} \\
\includegraphics[width=1\textwidth]{figure21}
\end{minipage}
}
\subfigure[the second subfigure]{
\begin{minipage}[b]{0.2\textwidth}
\includegraphics[width=1\textwidth]{figure1} \\
\includegraphics[width=1\textwidth]{figure1}
\end{minipage}
}
\end{figure}


\begin{lstlisting}[language=TeX,numbers=none,frame=lrtb,keywords={begin},label=Gamma,caption=Gamma]
\begin{figure}[H]
\centering
\subfigure[the first subfigure]{
\begin{minipage}[b]{0.2\textwidth}
\includegraphics[width=1\textwidth]{figure1} \\
\includegraphics[width=1\textwidth]{figure21}
\end{minipage}
}
\subfigure[the second subfigure]{
\begin{minipage}[b]{0.2\textwidth}
\includegraphics[width=1\textwidth]{figure1} \\
\includegraphics[width=1\textwidth]{figure1}
\end{minipage}
}
\end{figure}
\end{lstlisting}


\subsection{画表}


\begin{table}[H]
\centering
\caption{Comparison of algorithm efficiency between AINGA and SIGA}
\begin{tabular}{c|c|c}
\hline
Algorithm & Average steps & Success rate /\%\\\hline
AINGA & 92 & 100\\\hline
SIGA & 213 & 60\\\hline
\end{tabular}
\end{table}


\begin{lstlisting}[language=TeX,numbers=none,frame=lrtb,keywords={begin},label=Gamma,caption=Gamma]
\begin{table}[H]
\centering
\caption{Comparison of algorithm efficiency between AINGA and SIGA}
\begin{tabular}{c|c|c}
\hline
Algorithm & Average steps & Success rate /\%\\\hline
AINGA & 92 & 100\\\hline
SIGA & 213 & 60\\\hline
\end{tabular}
\end{table}
\end{lstlisting}



\begin{table}[H]
\centering
\caption{Comparison of tackle success rate between the team based on AINGA and the team based on SGA.}
\begin{tabular}{c|c|c|c|c}
\hline
Team & \tabincell{c}{50\\matches} & \tabincell{c}{100\\matches} & \tabincell{c}{200\\matches} & \tabincell{c}{Success \\rate/\%}  \\ \hline
AINGA & 89\% & 91\% & 90\% & 90\% \\\hline
Q-Learning & 84\% & 86\% & 83\% & 84.3\% \\\hline
\end{tabular}
\end{table}

\begin{lstlisting}[language=TeX,numbers=none,frame=lrtb,keywords={begin},label=Gamma,caption=Gamma]
\begin{table}[H]
\centering
\caption{Comparison of tackle success rate between the team based on AINGA and the team based on SGA.}
\begin{tabular}{c|c|c|c|c}
\hline
Team & \tabincell{c}{50\\matches} & \tabincell{c}{100\\matches} & \tabincell{c}{200\\matches} & \tabincell{c}{Success \\rate/\%}  \\ \hline
AINGA & 89\% & 91\% & 90\% & 90\% \\\hline
Q-Learning & 84\% & 86\% & 83\% & 84.3\% \\\hline
\end{tabular}
\end{table}
\end{lstlisting}

\section{参考文献}
\begin{lstlisting}[language=TeX,numbers=none,frame=lrtb,keywords={begin},label=Gamma,caption=Gamma]
@article{name1,
author = {作者, 多个作者用 and 连接},
title = {标题},
journal = {期刊名},
volume = {卷20},
number = {页码},
year = {年份},
abstract = {摘要, 这个主要是引用的时候自己参考的, 这一行不是必须的}
}

@book{name2,
author ="作者",
year="年份2008",
title="书名",
publisher ="出版社名称"
}
\end{lstlisting}

最简单的使用方法


\begin{lstlisting}[language=TeX,numbers=none,frame=lrtb,keywords={begin},label=Gamma,caption=Gamma]
\bibitem{CK1}Shing-jun Ren and Hai-de Huang (2001). “A Robot Path
Planning Algorithm Based On Grid Expansion.” Journal of
Harbin Institute of Technology. Vol.9,No. 11, pp. 68-72.
\end{lstlisting}

\begin{thebibliography}{11}
\bibitem{CK1}Shing-jun Ren and Hai-de Huang (2001). “A Robot Path
Planning Algorithm Based On Grid Expansion.” Journal of
Harbin Institute of Technology. Vol.9,No. 11, pp. 68-72.
\bibitem{CK2} Hong-yan Shi, Chang-zhi Sun et al. (2006). “Chaotic
Potential Field Method and Application in RobotSoccer
Game.” Proceedings of the 6th World Congress on
Intelligent Control andAutomation, Dalian, China, June
2006.

\end{thebibliography}
\end{document}

