数字与普通运算符号可直接由键盘上键入。下列符号可以直接由键盘键入:

        \begin{center}
                    + \;-\; =\; $<$\; $>$ \;/ \;:\; !\; | \;[\; ] \;(\; )\\
        \end{center}
		
要注意的是, 左右大括号$\{$ $\}$ 在 \LaTeX\ 中有特殊用途。欲排版左大括号, 指令为 $\backslash \{$ ,
右大括号之指令为 $\backslash \}$ 。排版展示数式有以下四种方法可以达到目的:
        \begin{center}
$\backslash$begin$\{$equation$\}$ ... $\backslash$end$\{$equation$\}$\\
$\backslash$begin$\{$displaymath$\}$ ... $\backslash$end$\{$displaymath$\}$\\
$\backslash$[ ... $\backslash$]\\
\$\$ ... \$\$
        \end{center}
除第一种方式外,其余将不对数学式子进行编号。数式内若要排版文字时,必须置于
$\backslash$mbox 指令内,否则将被视为数学符号,譬如,

$$f(x)=x^2-3x+1 \mbox{, where} -2 \leq x \leq 2$$
\section{常见的数学式}
本节列举一些常见的数学式作为练习与未来使用的参考,每个函数都有其特别之处,请仔细观察研究。
读者可以依此为基础,在往后的写作过程中,逐渐累积更多有特殊型态的或符号的数学式,
只要这里出现过的,参照原使档一定写得出来。

\subsection{函数}
    \begin{lstlisting}[language=TeX,numbers=none,frame=lrtb,keywords={begin},label=Binomial,caption=Binomial] 
  $f(x)={n\choose x}p^x(1-p)^{1-x}, \;\; x=0,1,2,\cdots,n$ 
  \end{lstlisting}
  $f(x)={n\choose x}p^x(1-p)^{1-x}, \;\; x=0,1,2,\cdots,n$ 
   
  \begin{lstlisting}[language=TeX,numbers=none,frame=lrtb,keywords={begin},label=Poisson,caption=Poisson] 
  $f(x)=\frac{e^{-\lambda}\lambda^x}{x!}, \;\;  x=0,1,2,\cdots$ 
  \end{lstlisting}
  $f(x)=\frac{e^{-\lambda}\lambda^x}{x!}, \;\;  x=0,1,2,\cdots$
  
  \begin{lstlisting}[language=TeX,numbers=none,frame=lrtb,keywords={begin},label=Gamma,caption=Gamma] 
  $f(x)=\frac{1}{\Gamma(\alpha)\beta^\alpha}x^{\alpha-1} e^{-\frac{x}{\beta}}, \;\; x\geq 0$
  \end{lstlisting}
  $f(x)=\frac{1}{\Gamma(\alpha)\beta^\alpha}x^{\alpha-1}e^{-\frac{x}{\beta}}, \;\; x\geq 0$ 
  
  \begin{lstlisting}[language=TeX,numbers=none,frame=lrtb,keywords={begin},label=Normal,caption=Normal] 
  $f(x)=\frac{1}{\sigma\sqrt{2\pi}}e^{-\frac{(x-\mu)^2}{2\sigma^2}}, \;\;  -\infty < x < \infty $
  \end{lstlisting}
  $f(x)=\frac{1}{\sigma\sqrt{2\pi}}e^{-\frac{(x-\mu)^2}{2\sigma^2}}, \;\;  -\infty < x < \infty $
  
    \begin{lstlisting}[language=TeX,numbers=none,frame=lrtb,keywords={begin},label=Int,caption=\song 积分式与方程式编号] 
  \begin{equation}\label{gamma}%..........label后的名称自订,代表该方程式
  \int^\infty_0 x^{\alpha-1}e^{-\lambda x} dx = \frac{\Gamma(\alpha)}{\lambda^{\alpha}}
  \end{equation}
  \end{lstlisting}
  \begin{equation}\label{gamma}%.................label后的名称自订,代表该方程式
  \int^\infty_0 x^{\alpha-1}e^{-\lambda x} dx = \frac{\Gamma(\alpha)}{\lambda^{\alpha}}
  \end{equation}
  
  方程式 (\ref{gamma})是广义 $\Gamma$ 积分。\footnote{\song这里利用方程式标签(label)来引用方程式,编号将自动更新。}
  
  \begin{lstlisting}[language=TeX,numbers=none,frame=lrtb,keywords={begin},label=Sqrt,caption=\song 开根号] 
  $$f(x)=\sqrt[3]{\frac {\displaystyle 4-x^{3}}{\displaystyle 1+x^{2}}}$$
  \end{lstlisting}
  $$f(x)=\sqrt[3]{\frac {\displaystyle 4-x^{3}}{\displaystyle 1+x^{2}}}$$
  
  \begin{lstlisting}[language=TeX,numbers=none,frame=lrtb,keywords={begin},label=limit,caption=\song 微分与极限(注意大刮号的使用)] 
  $$f'(x)=\frac{df(x)}{dx}=\lim_{h\rightarrow 0} \left( \frac{f(x+h)-f(x)}{h} \right)$$
  \end{lstlisting}  
  $$f'(x)=\frac{df(x)}{dx}=\lim_{h\rightarrow 0}\left(\frac{f(x+h)-f(x)}{h}\right)$$
  
    \begin{lstlisting}[language=TeX,numbers=none,frame=lrtb,keywords={begin},label=upanddown,caption=\song 上下限的使用] 
 $$\int_a^b f(x) dx \approx \lim_{n\rightarrow \infty}\sum_{k=1}^n f(x_k)\triangle x_k$$
  \end{lstlisting} 
  $$\int_a^b f(x) dx \approx \lim_{n\rightarrow \infty}\sum_{k=1}^n f(x_k)\triangle x_k$$
  
  \begin{lstlisting}[language=TeX,numbers=none,frame=lrtb,keywords={begin},label=bast,caption=\song 最佳化问题] 
   $$\max_{\mathbf{u},\mathbf{u}^T\mathbf{u}=1} \mathbf{u}^T\Sigma_X\mathbf{u}$$
  \end{lstlisting} 
  $$\max_{\mathbf{u},\mathbf{u}^T\mathbf{u}=1} \mathbf{u}^T\Sigma_X\mathbf{u}$$
  
  \begin{lstlisting}[language=TeX,numbers=none,frame=lrtb,keywords={begin},label=somesymbles,caption=\song 几个符号]
  $$\mathbf{e}=\mathbf{x}-\mathbf{x}_q=(I-P)\mathbf{x} \in V^{\perp}, \mbox{where}\; V\oplus V^{\perp}=R^p $$
  \end{lstlisting} 
  $$\mathbf{e}=\mathbf{x}-\mathbf{x}_q=(I-P)\mathbf{x} \in V^{\perp}, \mbox{where}\; V\oplus V^{\perp}=R^p $$


\subsection{矩阵与行列式}
矩阵或有规则排列的数学式或组合很常见,以下列举几种模式,请特别注意其使用的标签及一些需要注意的小地方。譬如,
\begin{enumerate}%[a)]
\item 矩阵的左右括号需各别加上。
\item 横行各项之间是以 $\&$ 区隔。
\item 除最后一行外,每行之末则加上换行指令$\backslash$ $\backslash$。
\item 使用array指令时,须加上选项以控制每一直栏内各数字或符号要居中排列、靠左或靠右。
\end{enumerate}
范例与注意事项:
\begin{enumerate}
  \item 左右方框刮号的使用及各直栏的对齐方式:
        $$ A = \left[
            \begin{array}{clr}
                a+b & mnop  & xy \\
                a+b & pn    & yz \\
                b+c & mp    & xyz
            \end{array} \right] $$
\begin{lstlisting}[language=TeX,numbers=none,frame=lrtb,keywords={begin}]
						$$ A = \left[
						\begin{array}{clr}
							a+b & mnop  & xy \\
							a+b & pn    & yz \\
							b+c & mp    & xyz
						\end{array} \right] $$
\end{lstlisting} 

  \item 左右圆框刮号的使用及各式点状:
        $$ A=\left(
            \begin{array}{cccc}
                a_{11} & a_{12} & \cdots & a_{1n}\\
                a_{21} & a_{22} & \cdots & a_{2n}\\
                \vdots & \vdots & \ddots & \vdots\\
                a_{n1} & a_{n2} & \cdots & a_{nn}
            \end{array} \right) $$
\begin{lstlisting}[language=TeX,numbers=none,frame=lrtb,keywords={begin}]
				$$ A=\left(
				\begin{array}{cccc}
					a_{11} & a_{12} & \cdots & a_{1n}\\
					a_{21} & a_{22} & \cdots & a_{2n}\\
					\vdots & \vdots & \ddots & \vdots\\
					a_{n1} & a_{n2} & \cdots & a_{nn}
				\end{array} \right) $$
\end{lstlisting} 

  \item 排列整齐的符号:
        $$ \begin{array}{clr}\\
            a+b+c   & m+n & xy \\
            a+b     & p+n & yz \\
            b+c     & m-n & xz
        \end{array} $$
\begin{lstlisting}[language=TeX,numbers=none,frame=lrtb,keywords={begin}]
					$$ \begin{array}{clr}\\
						a+b+c   & m+n & xy \\
						a+b     & p+n & yz \\
						b+c     & m-n & xz
					\end{array} $$
\end{lstlisting}

    \item 等号对齐的函数组合(不编号)
        \begin{eqnarray*}
          b_1 &=& d_1+c_1 \\
          a_2 &=& c_2+e_2
        \end{eqnarray*}
\begin{lstlisting}[language=TeX,numbers=none,frame=lrtb,keywords={begin}]
						\begin{eqnarray*}
						  b_1 &=& d_1+c_1 \\
						  a_2 &=& c_2+e_2
						\end{eqnarray*}
\end{lstlisting}

    \item 等号对齐的函数组合(编号在最后一行)
        \begin{eqnarray}
\nonumber b_1 &=& d_1+c_1 \\
          a_2 &=& c_2+e_2
        \end{eqnarray}
\begin{lstlisting}[language=TeX,numbers=none,frame=lrtb,keywords={begin}]
					\begin{eqnarray}
						\nonumber b_1 &=& d_1+c_1 \\
						a_2 &=& c_2+e_2
					\end{eqnarray}
\end{lstlisting}

    \item 使用巨集 amsmath 的指令 align(控制编号在第一行)
        \begin{align}
            b_1 &= d_1+c_1\\
            a_2 &= c_2+e_2 \notag
        \end{align}
\begin{lstlisting}[language=TeX,numbers=none,frame=lrtb,keywords={begin}]
					\begin{align}
						b_1 &= d_1+c_1\\
						a_2 &= c_2+e_2 \notag
					\end{align}
\end{lstlisting}

    \item 两组数学式分别对齐
    \begin{align}
        \alpha_1 &= \beta_1+\gamma_1+\delta_1, &a_1 &= b_1+c_1\\
        \alpha_2 &= \beta_2+\gamma_2+\delta_2, &a_2 &= b_2+c_2
    \end{align}
\begin{lstlisting}[language=TeX,numbers=none,frame=lrtb,keywords={begin}]
		\begin{align}
			\alpha_1 &= \beta_1+\gamma_1+\delta_1, &a_1 &= b_1+c_1\\
			\alpha_2 &= \beta_2+\gamma_2+\delta_2, &a_2 &= b_2+c_2
		\end{align}
\end{lstlisting}

    \item 编号在中间(split指令环境)
        \begin{equation}
            \begin{split}
                \alpha_1 &= \beta_1+\gamma_1\\
                \alpha_2 &= \beta_2+\gamma_2
            \end{split}
        \end{equation}
\begin{lstlisting}[language=TeX,numbers=none,frame=lrtb,keywords={begin}]
				\begin{equation}
					\begin{split}
						\alpha_1 &= \beta_1+\gamma_1\\
						\alpha_2 &= \beta_2+\gamma_2
					\end{split}
				\end{equation}
\end{lstlisting}

    \item 只是居中对齐的数学式组(gather指令环境)
        \begin{gather}
        \alpha_1 + \beta_1\notag\\
        \alpha_2 + \beta_2 + \gamma_2\notag
        \end{gather}
\begin{lstlisting}[language=TeX,numbers=none,frame=lrtb,keywords={begin}]
				\begin{gather}
					\alpha_1 + \beta_1\notag\\
					\alpha_2 + \beta_2 + \gamma_2\notag
				\end{gather}
\end{lstlisting}

    \item 长数学式的表达(注意第二行加号的位置)
        \begin{align}
            y   &= x_1 + x_2 + x_3 \notag\\
                &\quad + x_4 + x_5
        \end{align}
\begin{lstlisting}[language=TeX,numbers=none,frame=lrtb,keywords={begin}]
				\begin{align}
					y   &= x_1 + x_2 + x_3 \notag\\
						&\quad + x_4 + x_5
				\end{align}
\end{lstlisting}
\end{enumerate}


\subsection{其他}

  $$X_{n} \stackrel{d}{\longrightarrow} X$$\\
\begin{lstlisting}[language=TeX,numbers=none,frame=lrtb,keywords={begin}]
			$$X_{n} \stackrel{d}{\longrightarrow} X$$
\end{lstlisting}

  $$\overbrace{X_{1} + \ldots + \underbrace{X_{15} + \ldots + X_{30}}}$$\\
\begin{lstlisting}[language=TeX,numbers=none,frame=lrtb,keywords={begin}]
$$\overbrace{X_{1} + \ldots + \underbrace{X_{15} + \ldots + X_{30}}}$$
\end{lstlisting}

  \begin{equation*}
    G = \left\{\begin{array}{l}
          CLASS\#1 \;\;\mbox{if} \;\; \hat{\beta}^T\bf{x} \leq 0 \\
          CLASS\#2 \;\;\mbox{if} \;\; \hat{\beta}^T\bf{x} > 0
        \end{array}\right.
  \end{equation*}\\
\begin{lstlisting}[language=TeX,numbers=none,frame=lrtb,keywords={begin}]
  \begin{equation*}
    G = \left\{\begin{array}{l}
          CLASS\#1 \;\;\mbox{if} \;\; \hat{\beta}^T\bf{x} \leq 0 \\
          CLASS\#2 \;\;\mbox{if} \;\; \hat{\beta}^T\bf{x} > 0
        \end{array}\right.
  \end{equation*}
\end{lstlisting}

以equation或align排版时,数学式会自动编上号码。文稿其他地方若要引述某数学式,
可先以$\backslash$label指令加上标签,再使用$\backslash$ref指令引述。
如此一来若排版文稿须反覆修改,使用$\backslash$label 与$\backslash$ref 指令可以「自动对焦」不会出错。
